\documentclass[a4paper]{ltjarticle}
\usepackage{lltjext}

% ==============================================
% ページ設定のレイアウト確認
% \{layout}
% 本文中で「\layout」を呼び出す
% ==============================================
% ページ設定
% documentclass「a4paper (210*297mm)」を基準とする
% 1inch = 72pt = 25.2mm
% 1pt = 0.35mm
\usepackage{geometry}
\usepackage{graphicx} % rotateboxを使用するために必要
\geometry{
    top=0mm,  % 上部の余白領域
    bottom=0mm,  % 下部の余白領域
    inner=0mm,  % 左部の余白領域
    outer=0mm,  % 右部の余白領域
    headheight=0mm,  % ヘッダーの領域
    headsep=0mm,  % ヘッダーと本文領域の間隔('headheight'と'headsep'を0にするなら'nohead'でも良い)
    % nohead,
    marginparwidth=0mm,  % Margin Notesの領域
}
\pagestyle{empty}  % ページ番号を非表示
% ==============================================
% 表紙・背表紙
\begin{document}
    % \layout
    \begin{LARGE}
        % 表紙
        \setlength{\unitlength}{1mm}  % 座標系の単位を設定
        \begin{picture}(209, 296)  % (x, y) = (横方向の大きさ, 縦方向の大きさ)
            \put(30, 240){\makebox(150, 15)[b]{令和6年度卒業論文}}
            \put(30, 236){\line(1, 0){150}}
            % 日本語タイトル
            \put(30, 220){\makebox(150, 15)[b]{%
                \shortstack[l]{CtoCシェアサイクルサービスにおける\\新しいマッチングモデルの提案と実装}%
            }}
            % 英語タイトル (中央揃え)
            \put(30, 200){\makebox(150, 15)[c]{%
                \shortstack[c]{Proposal and Implementation of a New Matching Model \\for CtoC Bike Share Service}%
            }}
            \put(30, 201){\line(1,0){150}}
            \put(27, 87){\makebox(150, 15)[b]{福井大学大学院\ \ 工学研究科}}
            \put(30, 78){\makebox(150, 15)[b]{知識社会基礎工学専攻\ \ 知能モデリング研究室}}
            \put(30, 69){\makebox(150, 15)[b]{風~折\ \ 晃~輝}}
            \put(30, 65){\line(1,0){150}}
            \put(27, 50){\makebox(150, 15){主指導教員\ \ 小~髙\ 知~宏}}
            \put(27, 40){\makebox(150, 15){副指導教員\ \ 黒~岩\ 丈~介}}
            \put(27, 30){\makebox(150, 15){副指導教員\ \ 高~橋\ 泰~岳}}
            \put(30, 20){\makebox(150, 15){協力職員\ \ 松~田\ 優~也}}
            \put(34, 10){\makebox(150, 15){協力職員\ \ 諏~訪\ い~ず~み}}
        \end{picture}
        % 背表紙
        \begin{picture}(209, 296)
            \put(10, 268){\pbox<t>{令和}}
            \put(11.5, 260){6}
            \put(10, 245){\pbox<t>{年度}}
            \put(0, 238){\line(1, 0){30}}
            \put(10, 84){%
                \rotatebox{90}{%
                    \shortstack[c]{CtoCシェアサイクルサービスにおける\\新しいマッチングモデルの提案と実装}%
                }%
            }
            \put(0, 70){\line(1, 0){30}}
            \put(10, 35){\pbox<t>{風折 晃輝}}
        \end{picture}
    \end{LARGE}
\end{document}
