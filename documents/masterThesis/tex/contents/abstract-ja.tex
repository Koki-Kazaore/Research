 % 梗概 (Ja)
\section*{梗概}
    \par 近年,欧米や中国を中心に,世界的にシェアサイクルサービスが急速に普及しており,持続可能な交通手段として大きな注目を集めている.これらのサービスは,利用者が好きなタイミングで好きな期間,自転車を利用できるという利便性を提供しており,都市部における交通渋滞の緩和や環境負荷の軽減に寄与している.
    \par しかし,既存のシェアサイクルサービスは,企業が運営するBtoC型モデルであるため,地方では駅前などの一部地域でしか展開されず,初期投資や維持コストの高さから普及が困難とされている.都市部においても,サービスの需要が特定の時間や場所に偏り,自転車の不均衡な分布が生じ,事業者は自転車の再配置に係るコスト増加に直面している.
    \par 一方で,地方でもシェアサイクルへの需要を確実に存在し,公共交通が不足する地域や高齢化が進む地域では自転車のニーズが高まっている.これらの課題を解決するため,個人の遊休自転車を活用するCtoC型シェリングエコノミーに焦点を当てると,消費者同士が直接資産を共有することによって,遊休資産の有効活用や柔軟なサービス提供,コスト削減などのメリットが期待できる.
    \par そこで,本研究では,シェアサイクルの領域にCtoCシェアリングエコノミーを応用することを目的とし,自転車割り当てモデルの構築やIoTスマートロックプロトタイプの提案,それらを統合するためのAPI構築を行なった.CtoC型のシェアリングサービスを実現するため,主に数理最適化ベースの自転車の割り当てモデルの定式化及び実装を行い,シェアサイクルの需要と相関があると考えられるタクシーの実際のトリップデータを用い,シミュレーションによるモデルの検証を行った.再配置コストを主な評価指標に設定し,複数の割り当てモデルと比較した結果,本研究で定式化・実装した割り当てモデルが最も有意な結果となった.
