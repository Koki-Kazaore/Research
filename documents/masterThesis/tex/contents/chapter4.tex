% 第4章
\section{実装(\textcolor{orange}{7\%})}
  \label{sec:実装}
    \par
  
  \subsection{スマートロックの実装(0\%)}
    \label{sec:スマートロックの実装}
      \par
  
      \subsubsection{プロトタイプ作成(0\%)}
        \label{sec:プロトタイプ作成}
          \par
          
      \subsubsection{動作確認(0\%)}
        \label{sec:動作確認}
          \par
          
  \subsection{数理最適化モデルの実装(0\%)}
    \label{sec:数理最適化モデルの実装}
      \par
      
      \subsubsection{ソルバーの選定(\textcolor{green}{100\%}))}
        \label{sec:ソルバーの選定}
          \par ソルバーの性能は,結果の質や計算時間に大きく影響するため,自転車割り当て問題を効率的かつ正確に解くためには適切なソルバーの選定が必要不可欠である.特に,CtoCシェアサイクルシステムにおいては,リアルタイム性やスケーラビリティが求められるため,最適なソルバー選定がシステム全体のパフォーマンスを左右する.
          
          \par 本研究で扱っている自転車割り当て問題の特徴としては,整数線形計画問題として定式化される点が最大の特徴として挙げられる.他にも,変数にはバイナリ値のみが含まれ,制約条件は線形である点も特徴として挙げられる.問題の規模感としては,シェアリングの対象となる自転車の数やユーザの需要などに応じて決定変数である二値変数行列の成分の数が大規模になる可能性も考えられる.そのような状況下においても最適な自転車をユーザに割り当てる必要があるため,解の精度は高く保たれる必要がある.さらに,リアルタイム性が求めらる点も考慮すると,迅速な計算処理を継続して行う必要もある.
          
          \par そこで,ソルバーを選定するにあたって,上記の要件を満たすための観点をいくつかまとめる.まず性能面において,大規模問題でも迅速に解を求められ,効率的なメモリ管理が可能であることや,整数線形計画問題に対応していることが求められる.また,利便性の面において,無料で利用可能であることや,アルゴリズムの実装の際に用いるOR-Toolsとの互換性も求められる.なお,OR-ToolsとはGoogle社から提供されている組み合わせ最適化向けのオープンソースソフトウェアであり,非常に広範な可能性のあるソリューションの中から問題に対する最適化ソリューションを見つけ出すことをサポートする\scalebox{0.7}{\cite{OR-Tools}}.
          
          \par 具体的なソルバーの例として,CPLEXやGurobiなどが挙げられる.CPLEXはIBM社から提供されている,混合整数計画法のための分散型並列アルゴリズムと,線形計画,混合整数計画などのための柔軟で高性能な数理計画法ソルバーである\scalebox{0.7}{\cite{CPLEX}}.GurobiはGurobi社から提供されているソルバーであり,並列処理を最大限活用するよう構築され,高度なMIPヒューリスティックアルゴリズムにより実現可能解を素早く求解可能である特徴を持つ\scalebox{0.7}{\cite{Gurobi}}.しかし、これらのソルバーはオープンソースではないが故に,ライセンス費用の面における制限が懸念される.
          
          \par オープンソースソルバーの例としては,GLPKやCBCなどが挙げられる.
          
          \par GLPKはモスクワ航空大学のAndrew Makhorin氏によって開発されたソルバーである.C言語で記述されており,コマンドラインまたはAPIを介して操作可能であり,CとJavaのAPIを提供している.CPLEXのような商用ソルバーと比較するとGLPKの速度は劣るものの,線形計画問題に対して有効なソルバーである\scalebox{0.7}{\cite{gearhart2013comparison}}.
          
          \par CBCはJohn Forrest氏らによって開発された,COIN-OR線形計画法を用いた混合整数計画問題を解くためのソルバーである.C++で書かれており,呼び出し可能なライブラリとしても,スタンドアロンの実行ファイルとしても利用可能である.様々なモデリングシステムやパッケージなどを通して,様々な方法で利用することができる\scalebox{0.7}{\cite{CBC}}.
          
          \par また,オープンソースソルバーの別の例としてSCIPも挙げられる.SCIPは,混合整数線形計画問題や混合整数比線形計画問題,さらには制約整数計画問題のソルバーとして設計された,制約整数計画ソルバーである\scalebox{0.7}{\cite{bolusani2024scip}}.SCIPソルバーは汎用性の高いフレームワークであり,問題のサイズを縮小する前処理や下界値を強化するカット生成,より良い上限値を与えるためのヒューリスティック解法などの機能をプラグインとして追加することがでる.これによってSCIPソルバーの利用者は問題に合わせてカスタマイズし,性能を向上させることができる\scalebox{0.7}{\cite{shinano2013CIPSolver}}.一方で,SCIPを利用することの欠点としては,高機能であるが故に複雑性が高く,ある程度の学習コストを要する点や,多数のパラメータを持ち,その設定によって性能が大きく変化するため,最適なパラメータを見つけるためのチューニングが難しい場合がある.
          
          \par 上記で述べた要件やソルバーの一長一短を鑑み,本研究ではSCIPを自転車割り当てにおける整数線形計画問題のソルバーとして選定する.選定理由の最も大きなポイントとしては,オープンソースのソルバーであり,利用するにあたってライセンス関連のコストを懸念する必要が無い点である.オープンソースソルバーとしてSCIP以外に挙げたGLPKやCBCと比較して計算速度が優位である点もSCIPを選定したポイントの1つである.さらに,アルゴリズムの実装の際に用いるOR-Toolsとの互換性が高い点も選定ポイントとして挙げられる.SCIPを利用する際のデメリットとして挙げた複雑性について,OR-Toolsを併用して利用した場合,比較的シンプルにアルゴリズムを実装することが可能となるため,SCIPの長所を最大限生かした実装を行えることが期待される.
          
      \subsubsection{アルゴリズムの実装(0\%)}
        \label{sec:アルゴリズムの実装}
          \par
          
      \subsubsection{テストケースと結果(0\%)}
        \label{sec:テストケースと結果}
          \par  
  
  \subsection{機械学習モデルの実装(0\%)}
    \label{sec:機械学習モデルの実装}
      \par 
      
      \subsubsection{データセットの準備(0\%)}
        \label{sec:データセットの準備}
          \par
          
      \subsubsection{特徴量の選択と前処理(0\%)}
        \label{sec:特徴量の選択と前処理}
          \par
          
      \subsubsection{モデルの学習と評価(0\%)}
        \label{sec:モデルの学習と評価}
          \par
          
  \subsection{API開発(0\%)}
    \label{sec:API開発}
      \par
      
      \subsubsection{APIエンドポイントの実装(0\%)}
        \label{sec:APIエンドポイントの実装}
          \par
          
      \subsubsection{ドキュメンテーションとテスト(0\%)}
        \label{sec:ドキュメンテーションとテスト}
          \par
