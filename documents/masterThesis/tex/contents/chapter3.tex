% 第3章
\section{CtoCシェアサイクルのサービス設計とビジネスモデル(0\%)}
  \label{sec:CtoCシェアサイクルのサービス設計とビジネスモデル}
  
  \subsection{CtoCシェアサイクルの概要(0\%)}
    \label{sec:CtoCシェアサイクルの概要}
      \par $\Box$ CtoCシェアリングエコノミーの特徴
      \par $\Box$ 既存のCtoCシェアサイクルサービスの事例
      
  \subsection{サービス設計(0\%)}
    \label{sec:サービス設計}
      \par $\Box$ ユーザーエクスペリエンス(UX)の設計
      \par $\Box$ 自転車の管理と運用フロー
      \par $\Box$ スマートロックの役割と機能設計

  \subsection{コア技術の活用(0\%)}
    \label{sec:コア技術の活用}
      \par $\Box$ 2.3.1 スマートロック技術
      \par $\Box$ 2.3.1.1 ハードウェア構成
      \par $\Box$ 2.3.1.2 セキュリティ機能
      \par $\Box$ 2.3.1.3 通信プロトコルと接続性
      \par $\Box$ 2.3.2 マッチングアルゴリズム
      \par $\Box$ 2.3.2.1 数理最適化手法
      \par $\Box$ 2.3.2.2 機械学習手法
      \par $\Box$ 2.3.2.3 アルゴリズムの統合と運用
      \par $\Box$ 2.3.3 API設計と実装
      \par $\Box$ 2.3.3.1 APIの役割と重要性
      \par $\Box$ 2.3.3.2 設計原則とアーキテクチャ
      \par $\Box$ 2.3.3.3 多都市展開への対応
      
  \subsection{ビジネスモデルの構築(0\%)}
    \label{sec:ビジネスモデルの構築}
      \par $\Box$ 収益モデルの検討
      \par $\Box$ コスト構造の分析
      \par $\Box$ 市場分析とターゲットユーザー
      \par $\Box$ 競合分析と差別化戦
      
  \subsection{法的・倫理的考慮事項(0\%)}
    \label{sec:法的・倫理的考慮事項}
      \par $\Box$ 規制とコンプライアンス
      \par $\Box$ ユーザーのプライバシー保護
