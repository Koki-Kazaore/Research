% 第3章
\section{CtoCシェアサイクルのサービス設計とビジネスモデル(\textcolor{orange}{4\%})}
  \label{sec:CtoCシェアサイクルのサービス設計とビジネスモデル}
  
  \subsection{CtoCシェアサイクルの概要(0\%)}
    \label{sec:CtoCシェアサイクルの概要}
      \par $\Box$ CtoCシェアリングエコノミーの特徴
      \par $\Box$ 既存のCtoCシェアサイクルサービスの事例
      
  \subsection{サービス設計(0\%)}
    \label{sec:サービス設計}
      \par $\Box$ ユーザーエクスペリエンス(UX)の設計
      \par $\Box$ 自転車の管理と運用フロー
      \par $\Box$ スマートロックの役割と機能設計
      \par $\Box$ マイクロサービスの設計

  \subsection{コア技術の活用(\textcolor{orange}{22\%})}
    \label{sec:コア技術の活用}
      \par $\Box$ 3.3.1 スマートロック技術
      \par $\Box$ 3.3.1.1 ハードウェア構成
      \par $\Box$ 3.3.1.2 セキュリティ機能
      \par $\Box$ 3.3.1.3 通信プロトコルと接続性
      \par $\Box$ 3.3.2 マッチングアルゴリズム
      \par $\Box$ 3.3.2.1 数理最適化手法
      \par $\Box$ 3.3.2.2 機械学習手法
      \par $\Box$ 3.3.2.3 アルゴリズムの統合と運用
      \par $\Box$ 3.3.3 API設計と実装
      \par $\Box$ \sout{3.3.3.1 APIの役割と重要性}
      \par APIはApplication Programing Interfaceの略であり,2つのソフトウェアコンポーネントが一連の定義とプロトコルを使用して相互通信を可能とするメカニズムである\scalebox{0.7}{\cite{WhatIsAnAPI}}.自転車のライドシェア事業を例に挙げると,ユーザの現在地や自転車の位置などのリアルタイムデータを必要とする際に,APIを使用することでこれらのデータを効率的に取得・統合することが可能となる.また,効率的なスケーリングも特徴的であり,ビジネスの成長に合わせてシステムを柔軟に拡張することも可能である.
      \par $\Box$ 3.3.3.2 設計原則とアーキテクチャ
      \par APIアーキテクチャは原則,リクエストを送信するクライアントと,レスポンスを送信するサーバーの観点から説明される.特にREST(Representational State Transfer)設計原則に従って設計されたAPIをREST APIと呼ぶ.REST APIはクライアントとサーバーの責務を分離することを促進し,コンポーネントの実装を簡素化することに貢献する.ステートレスな通信が行われ,各リクエストはリクエストを理解するために必要な全ての情報を含んでおり,サーバーに保存されるコンテキストを利用することはできない.セッション状態は完全にクライアント側に保持され,これによって可視性や信頼性,スケーラビリティが向上する.可視性は各リクエストを分離して監視できるため,信頼性は部分的な障害からの回復が容易になるため,スケーラビリティはサーバーが以前のリクエストに関する情報を保存する必要がないためそれぞれ向上する\scalebox{0.7}{\cite{fielding2000architectural}}.
      \par $\Box$ 3.3.3.3 CORSについて。
      \par $\Box$ 3.3.3.3 Dockerについて。
      \par $\Box$ 3.3.3.3 クラウドコンピューティングについて。
      \par $\Box$ 3.3.3.3 多都市展開への対応
      
  \subsection{ビジネスモデルの構築(0\%)}
    \label{sec:ビジネスモデルの構築}
      \par $\Box$ 収益モデルの検討
      \par $\Box$ コスト構造の分析
      \par $\Box$ 市場分析とターゲットユーザー
      \par $\Box$ 競合分析と差別化戦
      
  \subsection{法的・倫理的考慮事項(0\%)}
    \label{sec:法的・倫理的考慮事項}
      \par $\Box$ 規制とコンプライアンス
      \par $\Box$ ユーザーのプライバシー保護
