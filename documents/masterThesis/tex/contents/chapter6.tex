% 第6章
\section{評価と結果(\textcolor{orange}{15\%})}
  \label{sec:評価と結果}
    \par
  
  \subsection{マッチングモデルの評価(0\%)}
    \label{sec:マッチングモデルの評価}
      \par
  
      \subsubsection{数理最適化モデルの結果分析(0\%)}
        \label{sec:数理最適化モデルの結果分析}
          \par シミュレーションを行う.

      \subsubsection{テストケースと結果(\textcolor{green}{100\%})}
        \label{sec:テストケースと結果}
          \par これまで,\ref{sec:数理最適化モデルの実装}節にてモデルの定式化やソルバーへの実装手順を示し,\ref{sec:数理最適化モデルの結果分析}項にて実際のデータを用いたシミュレーションを行った.本項では,それら実装したモデルやシミュレーションの妥当性の確認や解の品質評価を目的として,入力に対する出力の結果が意図する結果となっているかどうかを,テストケース及びテストコードを用意し,機械的に検証する.
          \par 数理最適化モデルが妥当な性能を発揮するかどうかを確かめるため,大きく分けて3種類のテストケースを用意する.それぞれ,小規模・中規模・大規模なデータである.小規模データでは,2台の利用可能な自転車が配置されている状態に対してユーザリクエストが1つ存在する状況のテストケースである.中規模データでは,5台の利用可能な自転車が配置されている状態に対してユーザリクエストが3つ存在する状況のテストケースである.大規模データでは,20台の利用可能な自転車が配置されている状態に対してユーザリクエストが10存在する状況のテストケースである.
          \par テストケースとして作成する自転車とユーザリクエストは,テスト観点を満たすための最低限のパラメータを持たせる.自転車は,現在地・自転車オーナの位置・利用終了予定時刻をパラメータとして持つ.ユーザリクエストは,利用開始位置・利用終了位置・利用開始時間・利用終了時間をパラメータとして持つ.また,\ref{sec:数理最適化モデルの結果分析}項のシミュレーションとは分離したテストを行うことでより妥当性が保証されるため,ニューヨークではなく東京周辺に自転車を配置したテストデータとする.
          \par 小規模データに関しては,さらに細かくテストケースを分割する.割り当てが成功する正常系と割り当てが失敗する異常系,制約条件を明らかに満たしていない例外ケースの3パターンを用意する.割り当てが失敗する異常系のテストケースは,表\ref{tab:小規模テストデータA及びBの自転車}及び表\ref{tab:小規模テストデータAのユーザリクエスト}に示す通りであり,これをテストデータAとしている.東京駅と新宿駅に2台の利用可能な自転車が駐輪されている状態で,東京駅から新宿駅までのユーザリクエストが存在しているシチュエーションである.結果としてユーザに自転車が割り当てられなかった.
          \par 割り当てが成功する正常系のテストケースは,表\ref{tab:小規模テストデータA及びBの自転車}及び表\ref{tab:小規模テストデータBのユーザリクエスト}に示す通りであり,これをテストデータBとしている.自転車の状態に関してはテストデータAに同じであるが,ユーザは,新宿駅から渋谷駅に向かいたいとする場合である.結果として新宿駅に駐輪されている自転車が割り当てられた.
          \par ここで,テストデータA及びBに関して,目的関数のパラメータを操作することによって,いかようにもなることが想定される.そのため,テストコード上で厳密に解の一致を判定することは行わず,想定しないエラーが発生せずに処理が完了することをテストしている.また,テストケースをカスタマイズして実際の挙動を確認する意義も果たしている.ただし,制約条件を明らかに満たしていない例外ケースはその限りではない.その場合はユーザに自転車が割り当てられるべきではないため,その旨が完全一致することをテストコード上で表現する.
          \par 具体的には,自転車の配置を表\ref{tab:小規模例外テストデータの自転車}に,ユーザリクエストは表\ref{tab:小規模テストデータAのユーザリクエスト}に示した通りのシチュエーションである.ユーザの利用開始位置に対して,自転車が明らかに制約条件を満たさない遠い距離に駐輪されている状態や,利用終了予定時間が未来である状態をテストしている.
          \par 小規模テストデータAやBと同じ要領で中規模データや大規模データに関してもテストケースを作成し,テストコードで機械的にテストを行えるよう整備した.データが大きくなるため,小規模データの説明の際に示したような表は割愛するが,付録のソースコードで確認できるため参照されたい.全てのテストケースに対して,それぞれ成功した場合は「SUCCESS」,失敗した場合は「FAILURE」をターミナルに出力できるようテストコードを整えて実行した結果が図\ref{fig:割り当てモデルのテスト}に示す通りである.全てのテストにパスしている.

          \begin{table*}[t]
            \caption{小規模テストデータA及びBの自転車}
            \label{tab:小規模テストデータA及びBの自転車}
            \centering
            \begin{tabular}{|l|l|l|l|} \hline
              自転車の現在地 & 自転車オーナの位置 & 利用終了予定時刻 & 場所(参考) \\ \hline
              (35.6804, 139.7690) & (35.6804, 139.7690) & NaT & 東京駅 \\
              (35.6895, 139.6917) & (35.6895, 139.6917) & NaT & 新宿駅 \\ \hline
            \end{tabular}
          \end{table*}
          
          \begin{table*}[t]
            \caption{小規模テストデータAのユーザリクエスト}
            \label{tab:小規模テストデータAのユーザリクエスト}
            \centering
            \begin{tabular}{|l|l|l|l|l|} \hline
              利用開始位置 & 利用終了位置 & 利用開始時間 & 利用終了時間 & 開始場所->終了場所(参考) \\ \hline
              (35.6804, 139.7692) & (35.6895, 139.6917) & now() & now()+15mins & 東京駅->新宿駅 \\ \hline
            \end{tabular}
          \end{table*}

          \begin{table*}[t]
            \caption{小規模テストデータBのユーザリクエスト}
            \label{tab:小規模テストデータBのユーザリクエスト}
            \centering
            \begin{tabular}{|l|l|l|l|l|} \hline
              利用開始位置 & 利用終了位置 & 利用開始時間 & 利用終了時間 & 開始場所->終了場所(参考) \\ \hline
              (35.6895, 139.6917) & (35.6618, 139.7012) & now() & now()+15mins & 新宿駅->渋谷駅 \\ \hline
            \end{tabular}
          \end{table*}

          \begin{table*}[t]
            \caption{小規模例外テストデータの自転車}
            \label{tab:小規模例外テストデータの自転車}
            \centering
            \begin{tabular}{|l|l|l|l|} \hline
              自転車の現在地 & 自転車オーナの位置 & 利用終了予定時刻 & 場所(参考) \\ \hline
              (35.8, 139.8) & (35.8, 139.8) & now()+1day & 足立区内某所 \\
              (35.81, 139.81) & (35.81, 139.81) & now()+1day & 埼玉県草加市某所 \\ \hline
            \end{tabular}
          \end{table*}

          \begin{figure}[htbp]
            \centering
            \includegraphics[scale=0.29]
            {figures/TestResult.png}
            \caption{割り当てモデルのテスト}
            \label{fig:割り当てモデルのテスト}
          \end{figure}

          
      \subsubsection{複数モデルの比較検討(0\%)}
        \label{sec:複数モデルの比較検討}
          \par
      
  \subsection{APIの適用性評価(0\%)}
    \label{sec:APIの適用性評価}
      \par
      
      \subsubsection{都市への展開シミュレーション(0\%)}
        \label{sec:都市への展開シミュレーション}
          \par $\Box$ 統合テスト
          \par $\Box$ エンドツーエンド(E2E)テスト
          
      \subsubsection{スケーラビリティとパフォーマンス評価(0\%)}
        \label{sec:スケーラビリティとパフォーマンス評価}
          \par $\Box$ 負荷テスト
