% 第1章
\section{はじめに}
  \label{sec:はじめに}
    \par 近年,欧米や中国を中心に,世界的にシェアサイクルサービスが急速に普及しており,持続可能な交通手段として大きな注目を集めている.これらのサービスは,利用者が好きなタイミングで好きな期間,自転車を利用できるという利便性を提供しており,都市部における交通渋滞の緩和や環境負荷の軽減に寄与している.実際に,世界の主要都市ではシェアサイクルの導入により,大気汚染の削減や健康増進効果も報告されており,その社会的・経済的な効果は無視できないものとなっている.
    \par しかし,既存のシェアサイクルサービスは主に企業が運営するBtoC(Business to Consumer)型モデルであり,その運用にはいくつかの課題が存在している.例えば,都市部と比較して人口の少ない地方では,駅前などの比較的栄えている一部の地域でのみサービスが展開されている場合が多く,地方全体としてはまだまだ普及していないという現状がある.その原因として,事業者がステーションやポートを設置する必要があり,その初期投資や維持管理コストが高額になるため,利用者数の少ない地域では採算が取れず,サービス展開が困難であることが挙げられる.結果として,需要の高い場所にステーションが集中し,郊外やニッチな需要に対応できていない状況が生まれている.
     \par また,都市部においても,シェアサイクルの利用は特定の時間帯や場所に偏りがちであり,それに伴い自転車の不均衡な分布が発生している.具体的には,通勤・通学時間帯には駅周辺の自転車が不足し,逆にオフィス街や学校周辺では自転車が過剰になるといった現象が見られる.これにより,必要な場所に自転車がない,あるいは不要な場所に自転車が集中するといった問題が生じている.事業者はこの不均衡を解消するために,自転車の再配置(リバランス)を行う必要があり,これには追加のコストと労力がかかる.このコスト増加は,サービス料金に反映される可能性があり,利用者の負担増加やサービス品質の低下につながる懸念がある.
      \par 一方,地方部においてもシェアサイクルサービスを利用したいという需要は確実に存在していると考えられる.例えば,公共交通機関が不足している地域では,最寄りのバス停や駅までのアクセス手段として自転車のニーズが高い.また,高齢化が進む地域では,高齢者の移動手段として電動アシスト自転車の需要が増えており,これらのニーズに対応できるシェアサイクルサービスの提供が望まれている.
       \par このようなBtoC型シェアサイクルの課題を解決するために,個人が所有する遊休自転車を有効活用するCtoC(Consumer to Consumer)型シェアリングエコノミーが注目されている.CtoC型シェアリングエコノミーは,消費者同士が直接的に資産を共有することを特徴とし,遊休資産の活用や柔軟なサービス提供,初期投資や運営コストの削減といったメリットが期待される.特に,シェアサイクルの分野においても,CtoC型モデルを導入することで既存のBtoC型モデルが抱える課題を解決できる可能性がある.
       \par 具体的には,自転車を借りる立場からのシェアサイクルだけでなく,自転車を貸す立場からのシェアサイクルにも一定数の需要があると考えられる.例えば,自転車を所有しているが,普段ほとんど使用することがなく,駐輪場や自宅で眠っている自転車を有効活用したいと考える人々がいる.これらの遊休自転車をシェアサイクルサービスの一部として他人に貸し出すことで,所有者は副収入を得ることができ,利用者は必要なときに手軽に自転車を借りることができる.このような相互に利のある関係を構築することで,自転車の有効活用が促進される.
       \par さらに,使われていない自転車が増加することによる駐輪場の不足や放置自転車の問題の解決にもつながる可能性がある.環境面でも,資源の有効活用や廃棄物の削減といった効果が期待できる.加えて,個人所有の自転車は都市部と地方の双方に存在しており,その分布は比較的均一であるため,地域の偏りなくサービスを展開できる可能性が高い.
        \par あらゆる地域に散在している個人所有の自転車をモビリティのリソースとして有効活用できれば,交通手段の多様化と効率化が促進され,地方における交通弱者の支援や,観光地での移動手段の提供など,多面的な効果が期待できる.また,都市交通の混雑緩和や環境負荷の軽減にも寄与し,持続可能な社会の実現に貢献できるのではないだろうか.
        \par しかし,CtoC型シェアサイクルを実現するためには,いくつかの課題も存在する.例えば,貸し手と借り手の間での信用問題や,自転車のメンテナンス・安全性の確保,事故発生時の保険対応などが挙げられる.これらの課題を克服するためには,適切なプラットフォームの構築や法律・規制の整備,そしてユーザー教育が必要である.
        \par そこで本研究では,、このCtoC型シェアサイクルに着目し,その可能性を探求し,その導入に向けた課題と解決策を検討する.具体的には,既存のシェアサイクルサービスの分析からユーザー調査,プラットフォームの設計・開発,および実際のデータを用いたシミュレーション実験を通じて,CtoC型モデルの有効性と実現に向けたきっかけとなることを目的とする.この研究により,持続可能な交通手段としてのシェアサイクルサービスの更なる発展に寄与し,地域社会の活性化や環境問題の解決に貢献したいと考える.
        \par 前述したように,地方でシェアサイクルサービスが普及しない原因として採算性が悪い点が考えられる.しかし,個人所有の自転車をシェアする場合については,予めシェアするための自転車を用意する必要が無いため,初期費用がかなり抑えられることが期待される.また,個人所有の自転車を駐輪している駐輪場が,一般的なシェアサイクルサービスにおけるポートのような存在になるため,新しくポートを確保するための手間や費用も抑えられる.なおかつ,あらゆる場所をポートととして捉えることができる.
        \par 加えて,個人所有の自転車をシェアするサービスを利用することで,自転車を貸す側にもインセンティブが入る.これは,自転車のオーナーが自転車を利用していない時間にお金が入ることを意味する.つまり,シェアサイクルサービス全体の初期費用が抑えられる上,自転車を借りる側にも貸す側にもメリットがあることとなる.価格設定に関しても自転車のオーナーとサービスのユーザー間での調整によって決定することも可能となるため,従来のシェアサイクルサービスに比べて,利用料金の自由度が高いことも特徴的な面になる.
        \par 研究アプローチとしては,まず,システムの利用者のニーズやシナリオ等を定義するユーザー要件や,システムの機能面における必要動作を定義する機能要件,システムの機能以外の面におけるシステムの振る舞いを定義する非機能要件に大別し,それぞれを明確に定義し,サービスにおけるビジネスやシステムの全体像の設計を行う.次に,自転車に取り付けるためのスマートロックのプロトタイプを開発する.ESP32マイコンやサーボモータ,NFCタグや指紋認証センサなどのハードウェアモジュールを用いて実装する.NFCと指紋認証に2つの認証方式を実装し,それぞれで解錠・施錠処理の実現を目指している.続いて,自転車の効率的な割り当てを実現するための整数線形計画法による数理最適化モデルを定式化し,数理最適化ベースの自転車割り当てモデルを構築する.ここでは,SCIPソルバーを用いてモデルを実装し,最適解を求める処理を担う.実際のニューヨーク市内のタクシートリップデータを用いたシミュレーション実験を行い,提案も出るの有効性を検証する.数理最適化ベースの割り当てモデルの他,ランダム割り当てモデルや最近傍割り当てモデル,逐次最適化割り当てモデルなどの複数の比較対象モデルも同時に構築し,提案モデルとの性能比較も行っている.構築した割り当てモデルを利用するためのインターフェースとしてAPIの開発とデプロイも実施している.PythonのFastAPIフレームワークを用いてREST APIを構築し,APIドキュメントの作成とテストを行うことで,APIの品質と信頼性を向上させることに努めた.また,Dockerを用いてAPIをコンテナ化し,クラウド環境にデプロイしている.
        \par 最後に,本論文の構成を説明する.本論文ではまず,\ref{sec:関連研究}章にて本研究の関連研究をまとめる.シェアサイクルやシェアリングエコノミー,スマートロックやマッチングアルゴリズム,API設計に関する既存の研究等をレビューし,本研究の立ち位置を明確にする.特に,BtoC型シェアサイクルの課題とCtoC型シェアリングエコノミーの動向の分析に注力している.\ref{sec:CtoCシェアサイクルのサービス設計とビジネスモデル}章では,本研究で提案するCtoC型シェアサイクルサービスの全体像やコンセプト,サービス設計やコア技術の活用,ビジネスモデルについて詳述する.ユーザーエクスペリエンス(UX)を考慮したペルソナ設計やユーザーシナリオについても説明する.\ref{sec:システム設計}章では,システム要件の定義や全体のアーキテクチャにはじまり,データフローやスマートロックの設計,数理最適化マッチングモデルやAPIの設計について説明する.この章にて機能要件や非機能要件についても定義する.\ref{sec:実装}章では,スマートロックプロトタイプから数理最適化モデルやAPIの実装まで,具体的な手順やコード例を説明する.利用したプログラミング言語やライブラリ,フレームワークについても言及する.\ref{sec:評価と結果}章では,数理最適化モデルの評価やシミュレーション結果,テストケースによる検証を行い,その結果をまとめる.\ref{sec:考察}章では,主に\ref{sec:評価と結果}章の結果を参照し,システム全体の有効性やCtoC化によるメリット・デメリットなど,その結果から得られる考察をまとめる.

