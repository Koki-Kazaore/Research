% 梗概 (En)
\section*{Abstract}
    \par In recent years, bike share services have rapidly gained popularity worldwide, especially in Europe, the United States, and China, attracting significant attention as a sustainable mode of transportation. These services offer users the convenience of renting bikes at any time and for any duration, contributing to alleviating traffic congestion in urban areas and reducing environmental impact.
    \par However, existing bike share services typically operate under a B2C (Business to Consumer) model run by companies. As a result, these services are often limited to specific locations, such as areas near train stations in rural regions, and their widespread adoption is hindered by high initial investment and maintenance costs. Even in urban areas, demand for these services tends to concentrate in specific times and locations, leading to an uneven distribution of bikes. This imbalance forces service providers to bear increased costs for bike redistribution.
    \par On the other hand, there is undoubtedly demand for bike share services in rural areas. In regions lacking adequate public transportation or with aging populations, the need for bikes is increasing. To address these challenges, focusing on a CtoC (Consumer to Consumer) sharing economy model that leverages unused personal bikes offers potential benefits. By enabling consumers to directly share their assets, this model can promote the effective use of idle resources, provide more flexible service offerings, and reduce costs.
    \par In this study, we aim to apply the CtoC sharing economy model to the domain of bike share services. We have developed a bike allocation model, proposed a prototype IoT smart lock, and constructed APIs to integrate these components. To realize a CtoC sharing service, we formulated and implemented a mathematical optimization-based bike allocation model. Using real taxi trip data, which is considered to correlate with bike share demand, we validated the model through simulations. By setting relocation costs as the primary evaluation metric and comparing multiple allocation models, the allocation model formulated and implemented in this study yielded the most significant results.
