% 第2章
\section{関連研究(\textcolor{orange}{30\%})}
  \label{sec:関連研究}
    \par
  
  \subsection{シェアサイクルの現状と課題(\textcolor{green}{100\%})}
    \label{sec:シェアサイクルの現状と課題}
      \par シェアサイクルは,都市部を中心に世界各地で普及している持続可能な交通手段であり,そのビジネスモデルは主にBtoC型である.本節では,これらのモデルの現状と課題について検討し,特にBtoC型シェアサイクルの限界とCtoC型シェアリングエコノミーの動向に焦点を当てる.
      \par シェアサイクルの分野でも,CtoC型モデルを導入することによって,BtoC型のシェアサイクルビジネスモデルが抱える課題を解決し得る可能性がある.遊休自演者の有効活用や,需要に応じた柔軟なサービス提供,初期投資や運営コストの削減など,CtoC型ならではのメリットが期待される.
      
      \subsubsection{BtoC型シェアサイクルの限界(\textcolor{green}{100\%})}
        \label{sec:BtoC型シェアサイクルの限界}
          \par BtoC型(Business-to-Consumer)シェアサイクルとは,事業者が自転車を保有および運営し,利用者がそのサービスを利用して自転車を借りるというビジネスモデルである.BtoC型では,特定の企業や自治体が自転車を大規模に保有・管理し,ユーザが自転車が駐輪されているステーションや無人ポートから自転車を借りる.欧米やアジアなどの大都市を中心として世界的に普及しており,代表的なサービスに,日本のドコモ・バイクシェア,アメリカのCiti Bike,イギリスのSantander Cycles,フランスのVelib’などが挙げられる.
          \par しかし,BtoC型シェアサイクルにはBtoC型であるが故の課題を抱えている.特に初期投資や運営に係るコストの高さや,地理的・時間的再配置の難しさ,自転車ポートの配置や柔軟性・拡張性の制約などが挙げられる.
          \par 世界の40都市における自転車シェアリングシステムの空間的な分布,公共交通機関との統合,および全体的な有効性を分析したMahajan氏らの研究\scalebox{0.7}{\cite{bike-sharing-accessibility}}では,自転車シェアリングサービスのアクセシビリティを評価するための新しい指標であるBSAI(Bike-Share Service Accessibility Index)を導入し,シェアサイクルサービスの利用のしやすさやパフォーマンスを定量的に評価した.
          \par BSAIが高い都市の例として,トロントやワシントン,チューリッヒなどが挙げられている.これらの都市は,広範なサービス範囲や公共交通機関との統合,効率的なインフラ投資を行っているという特徴を持っている.一方で,BSAIが低い都市のの例として,オースティンやサンアントニオなどが挙げられている.これらの都市は,サービス範囲が限定している点や,公共交通機関との連携不足,非効率的なインフラ投資などの課題を持つ点に共通している.
          \par この研究が示唆している課題は,自転車シェアリングシステムのインフラが,都市のすべての住民に公平にアクセス可能ではない点にある.自転車ステーションやポートの分布には地域格差があり,特にBSAIが低い都市ではポートの設置範囲や数が限定されている.それにより,自転車ステーションやポートが特定のエリア内においてのみ利用可能であるため,ユーザのアクセシビリティが悪化している.また,公共交通機関との連携が不足している点も,サービス範囲が限定している点に起因すると考えられる.ただ,BtoC型シェアサイクルでは,企業は利益のために,需要の高い地域に自転車ステーションやポートを設置するはずであるため,必然的な課題とも捉えられる.設置エリア外への展開が難しく,郊外やニッチな需要には対応できない点もBtoC型シェアサイクルの課題である.
          \par また,自転車シェアリング事業の長期的な成功のための政策とビジネス上の教訓を提示することを目的としNikitas氏の研究\scalebox{0.7}{\cite{nikitas2019save}}では,スウェーデンとギリシャでのアンケート調査と,世界各地の自転車シェアリングの成功例と失敗例の分析を組み合わせ,その知見を導き出している.
          \par この研究では,既存の自転車シェアリング事業の成功と失敗の事例を詳細に分析し,その要因を特定している.例えば,アメリカで初めてドックレスバイクシェアシステムを提供した中国のバイクシェアリング企業Bluegogo社は,過剰な自転車供給と需要不足のため,2017年11月に経営破綻した.北京大学発のスタートアップOfo社も,英国事業が赤字となり,2019年1月には3,000台を保有していたロンドンから撤退することとなった.
          \par これらは論文で紹介されいてる事例の一部にすぎないが,この研究が示唆している課題に,過剰共有が挙げられる.ドックレスバイクシェアリングの急速な拡大に伴い,多くの都市で自転車が過剰供給されており,需要と供給のミスマッチが生じていると考えられる.しかし,BtoC型シェアサイクルにおいては,サービスを提供するために大規模な自転車を事前に用意する必要があるため,避けがたい課題である.
          \par さらに,地理的・時間的な再配置の難しさについても言及されている.ドックレスバイクシェアリングの導入によって,従来のステーションベースでは不可能であった乗り捨てによるドアツードアの利便性が向上した.その一方で,ユーザによって利用時間帯や場所が偏るため,自転車の分布が不均一になりやすく,需要のある場祖に自転車が不足する,または需要のない場所に自転車が過剰に集中するという課題が発生している.自転車が供給されないまま放置されていると,ユーザのアクセシビリティ低下を招くほか,景観を損ねるなどの,ユーザ以外への影響も懸念される.そのため,定期的に自転車の再配置を行う必要がある.しかし,再配置には人力・トラック等が必要であり,コスト増加やオペレーションの難化を招いている.

      \subsubsection{CtoC型シェアリングエコノミーの動向(\textcolor{green}{100\%})}
        \label{sec:CtoC型シェアリングエコノミーの動向}
          \par シェアサイクルのBtoC型モデルにて例示した課題を受け,様々な事業ドメインにおいてCtoC型シェアリングエコノミーが注目されている.本研究の研究領域であるCtoCシェアサイクルもCtoC型シェアリングエコノミーの一部として捉えることができるため,本項でCtoC型の特徴や動向をまとめ,俯瞰することとする.
          \par まず,CtoC型シェアリングエコノミーを定義するにあたり,Frenken氏らの研究\scalebox{0.7}{\cite{frenken2019putting}} が参考になる.この研究では,シェアリングエコノミーの定義を明確にし,その経済的,社会的,環境的影響を評価し,既存の規制と代替プラットフォームの構造について考察している.Frenken氏らは,シェアリングエコノミーを「消費者が,一時的に,余剰の物理的資産(遊休能力)を,場合によっては金銭を介して相互に利用し合うこと」と定義している.
          \par この定義は,従来の共有の概念とは異なり,特に消費者間の取引であることや遊休能力の活用などが強調されている.つまり,ここで定義しているシェアリングエコノミーは,企業を介さず,消費者同士が直接的に資産を共有し,その共有する資産は,所有者が常に使用しているわけではない余剰の資産(例えば,空いている部屋や使っていない自動車)が共有されることを意味している.ただし,企業はプラットフォーマーとして間接的に取引に介在する可能性はある.これはインターネットプラットフォームの登場により,「見知らぬ者同士の共有(stranger sharing)」という概念が生まれたためである.
          \par シェアリングエコノミーの性質,運営メカニズム,および伝統的な経済理論と市場運営への影響を経済学の観点から詳細に分析することを目的としたChen氏の研究\scalebox{0.7}{\cite{AnalysesAndPerspectivesFromAnEconomicPerspective}}では,具体的な事例分析を通してシェアリングエコノミーの実態を明らかにしている.CtoC型シェアリングエコノミーのプラットフォーム型ビジネスの例として,UberとAirbnbが挙げられる.Uberは,都市交通の分やでドライバーと乗客を結び付け,交通手段の効率化と柔軟な移動オプションを提供し,Airbnbは,住宅所有者が空き部屋を旅行者に貸し出すことを可能にし,宿泊業界に新たな選択肢をもたらした.このような事例から,CtoC型のシェアリングエコノミーおよびそのプラットフォームは,遊休資産を有効活用することで新たな価値を創出し,消費者の行動や産業構造を変化させ,今日の隆盛を迎えるまで成長を遂げてきた.
          \par このCtoC型シェアリングエコノミーの成長と促進の要因は,技術や社会,経済など多岐にわたる.
          \par 協調消費(Collaborative Consumption)への参加を促す動機を調査し,持続可能性や楽しさ,評判や経済的利益といった要因が,協調消費に対する態度や行動意図にどのように影響するかを分析したHamari氏らの研究\scalebox{0.7}{\cite{hamari2016sharing}}では,技術や社会,経済のそれぞれの観点からその要因について言及されている.
          \par 情報通信技術の観点においては,Web2.0の発展により,ユーザ生成コンテンツの増加やオンラインでの共同作業が容易になった点やオンラインプラットフォームを通じて物品やサービスの共有が促進された点,またそれらによってオープンソースソフトウェアやP2Pファイル共有などの様々な形態のシェアリングエコノミーが生まれたことがシェアリングエコノミーの成長要因として挙げられる.
          \par 経済的観点においては,所有することよりもアクセスできることを重視し,必要な時に必要な分だけ利用することに対する消費者の価値観の変化がシェアリングエコノミーの成長を後押ししている.また,CtoCプラットフォームは,従来のサービスよりも安価にサービスを提供できることが多く,消費者にとって魅力的な選択肢となる.例えば,Airbnbはほてるよりも安価な宿泊施設を提供し,Uberはタクシーよりも手頃な価格で移動手段を提供できる場合がある.また,これはユーザだけでなく,提供者としての個人が自分の遊休資産を貸し出すことによって収入を得る機会を提供することにもなる.これによって個人が新たな収入源を確保し,経済的な自立を促進できることも成長要因として挙げられる\scalebox{0.7}{\cite{sundararajan2017sharing}}.
          \par 社会的観点においては,オンラインプラットフォームにおけるユーザレビューや評価システムが,見知らぬ人同士の取引における信頼を構築する上で重要な役割を果たし,個人間でもスムーズに取引が可能となった点が成長要因として挙げられる.また,環境問題意識への関心の高まりから,環境負荷を低減するために持続可能な消費を求める消費者が増加している点も起因していると考えられる.
          \par 本研究とは異なる事業ドメインで構築されているシェアリングエコノミーから着想を得ることで,シェアサイクルの事業ドメインでも応用可能性があると考えられる.
      
  \subsection{スマートロック技術の進展(0\%)}
    \label{sec:スマートロック技術の進展}
      \par
      
      \subsubsection{ハードウェアの最新技術(0\%)}
        \label{sec:ハードウェアの最新技術}
          \par
          
      \subsubsection{セキュリティと信頼性の確保(0\%)}
        \label{sec:セキュリティと信頼性の確保}
          \par  
  
  \subsection{マッチングアルゴリズムの研究動向(0\%)}
    \label{sec:マッチングアルゴリズムの研究動向}
      \par 
      
      \subsubsection{数理最適化によるマッチング手法(0\%)}
        \label{sec:数理最適化によるマッチング手法}
          \par
          
      \subsubsection{機械学習によるマッチング手法(0\%)}
        \label{sec:機械学習によるマッチング手法}
          \par    
          
  \subsection{API設計と都市への適用性(0\%)}
    \label{sec:API設計と都市への適用性}
      \par 
