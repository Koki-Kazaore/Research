% 第2章
\section{関連研究(\textcolor{orange}{10\%})}
  \label{sec:関連研究}
    \par
  
  \subsection{シェアサイクルの現状と課題(0\%)}
    \label{sec:シェアサイクルの現状と課題}
      % \par 「2.1 BtoC型シェアサイクルの現状と課題」の中で、この節では「BtoC型」モデルの特徴とその限界について焦点を当てることを明示します。
      
      \subsubsection{BtoC型シェアサイクルの限界(\textcolor{green}{100\%})}
        \label{sec:BtoC型シェアサイクルの限界}
          \par BtoC型(Business-to-Consumer)シェアサイクルとは,事業者が自転車を保有および運営し,利用者がそのサービスを利用して自転車を借りるというビジネスモデルである.BtoC型では,特定の企業や自治体が自転車を大規模に保有・管理し,ユーザが自転車が駐輪されているステーションや無人ポートから自転車を借りる.欧米やアジアなどの大都市を中心として世界的に普及しており,代表的なサービスに,日本のドコモ・バイクシェア,アメリカのCiti Bike,イギリスのSantander Cycles,フランスのVelib’などが挙げられる.
          \par しかし,BtoC型シェアサイクルにはBtoC型であるが故の課題を抱えている.特に初期投資や運営に係るコストの高さや,地理的・時間的再配置の難しさ,自転車ポートの配置や柔軟性・拡張性の制約などが挙げられる.
          \par 世界の40都市における自転車シェアリングシステムの空間的な分布,公共交通機関との統合,および全体的な有効性を分析したMahajan氏らの研究\scalebox{0.7}{\cite{bike-sharing-accessibility}}では,自転車シェアリングサービスのアクセシビリティを評価するための新しい指標であるBSAI(Bike-Share Service Accessibility Index)を導入し,シェアサイクルサービスの利用のしやすさやパフォーマンスを定量的に評価した.
          \par BSAIが高い都市の例として,トロントやワシントン,チューリッヒなどが挙げられている.これらの都市は,広範なサービス範囲や公共交通機関との統合,効率的なインフラ投資を行っているという特徴を持っている.一方で,BSAIが低い都市のの例として,オースティンやサンアントニオなどが挙げられている.これらの都市は,サービス範囲が限定している点や,公共交通機関との連携不足,非効率的なインフラ投資などの課題を持つ点に共通している.
          \par この研究が示唆している課題は,自転車シェアリングシステムのインフラが,都市のすべての住民に公平にアクセス可能ではない点にある.自転車ステーションやポートの分布には地域格差があり,特にBSAIが低い都市ではポートの設置範囲や数が限定されている.それにより,自転車ステーションやポートが特定のエリア内においてのみ利用可能であるため,ユーザのアクセシビリティが悪化している.また,公共交通機関との連携が不足している点も,サービス範囲が限定している点に起因すると考えられる.ただ,BtoC型シェアサイクルでは,企業は利益のために,需要の高い地域に自転車ステーションやポートを設置するはずであるため,必然的な課題とも捉えられる.設置エリア外への展開が難しく,郊外やニッチな需要には対応できない点もBtoC型シェアサイクルの課題である.
          \par また,自転車シェアリング事業の長期的な成功のための政策とビジネス上の教訓を提示することを目的としNikitas氏の研究\scalebox{0.7}{\cite{nikitas2019save}}では,スウェーデンとギリシャでのアンケート調査と,世界各地の自転車シェアリングの成功例と失敗例の分析を組み合わせ,その知見を導き出している.
          \par この研究では,既存の自転車シェアリング事業の成功と失敗の事例を詳細に分析し,その要因を特定している.例えば,アメリカで初めてドックレスバイクシェアシステムを提供した中国のバイクシェアリング企業Bluegogo社は,過剰な自転車供給と需要不足のため,2017年11月に経営破綻した.北京大学発のスタートアップOfo社も,英国事業が赤字となり,2019年1月には3,000台を保有していたロンドンから撤退することとなった.
          \par これらは論文で紹介されいてる事例の一部にすぎないが,この研究が示唆している課題に,過剰共有が挙げられる.ドックレスバイクシェアリングの急速な拡大に伴い,多くの都市で自転車が過剰供給されており,需要と供給のミスマッチが生じていると考えられる.しかし,BtoC型シェアサイクルにおいては,サービスを提供するために大規模な自転車を事前に用意する必要があるため,避けがたい課題である.
          \par さらに,地理的・時間的な再配置の難しさについても言及されている.ドックレスバイクシェアリングの導入によって,従来のステーションベースでは不可能であった乗り捨てによるドアツードアの利便性が向上した.その一方で,ユーザによって利用時間帯や場所が偏るため,自転車の分布が不均一になりやすく,需要のある場祖に自転車が不足する,または需要のない場所に自転車が過剰に集中するという課題が発生している.自転車が供給されないまま放置されていると,ユーザのアクセシビリティ低下を招くほか,景観を損ねるなどの,ユーザ以外への影響も懸念される.そのため,定期的に自転車の再配置を行う必要がある.しかし,再配置には人力・トラック等が必要であり,コスト増加やオペレーションの難化を招いている.
          
      \subsubsection{CtoC型シェアリングエコノミーの動向(0\%)}
        \label{sec:CtoC型シェアリングエコノミーの動向}
          \par 
      
  \subsection{スマートロック技術の進展(0\%)}
    \label{sec:スマートロック技術の進展}
      \par
      
      \subsubsection{ハードウェアの最新技術(0\%)}
        \label{sec:ハードウェアの最新技術}
          \par
          
      \subsubsection{セキュリティと信頼性の確保(0\%)}
        \label{sec:セキュリティと信頼性の確保}
          \par  
  
  \subsection{マッチングアルゴリズムの研究動向(0\%)}
    \label{sec:マッチングアルゴリズムの研究動向}
      \par 
      
      \subsubsection{数理最適化によるマッチング手法(0\%)}
        \label{sec:数理最適化によるマッチング手法}
          \par
          
      \subsubsection{機械学習によるマッチング手法(0\%)}
        \label{sec:機械学習によるマッチング手法}
          \par    
          
  \subsection{API設計と都市への適用性(0\%)}
    \label{sec:API設計と都市への適用性}
      \par 
