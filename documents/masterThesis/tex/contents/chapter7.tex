% 第7章
\section{考察(0\%)}
  \label{sec:考察}
    % \par 「6.1.2 テストケースと結果」の小規模データ異常系の話については、割り当てられる方が自然かも。ここは重みを調整することで対応可能と考えられる。
    % \par fig:自転車の割り当て成功率とステータスの変化 から分かる通り,割り当て成功率が高くない.なぜならNYCの7万以上のタクシーデータに対して10台の自転車しか用意していないから.それが原因でfig:モデル別再配置コストの時間経過比較 もあまり変化していないのではないだろうか.否,自転車を50台にスケールアップ(fig:50台の時の再配置コストの時間経過比較)してみるとより顕著な差が出ることがわかった。
  
  \subsection{システム全体の有効性(0\%)}
    \label{sec:システム全体の有効性}
      \par
      
  \subsection{CtoC化によるメリットとデメリット(0\%)}
    \label{sec:CtoC化によるメリットとデメリット}
      \par
      
  \subsection{技術的課題と解決策(0\%)}
    \label{sec:技術的課題と解決策}
      \par
      
  \subsection{社会的インパクトと倫理的考慮(0\%)}
    \label{sec:社会的インパクトと倫理的考慮}
      \par
