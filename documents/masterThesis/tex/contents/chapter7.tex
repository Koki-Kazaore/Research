% 第7章
\clearpage
\newpage

\section{考察}
  \label{sec:考察}
    \par 本研究では,個人所有の自転車を対象としたCtoC型のドックレスシェアサイクルシステムを提案し,その有効性について検証を行った.以下では,システム全体の有効性,CtoC化によるメリットとデメリット,技術的課題と解決策,そして社会的インパクトと倫理的考慮について総合的に考察する.
    \par また,本研究は限定的な実験環境下で評価を行ったに過ぎないため,現場での利用状況や利用者からのフィードバックを踏まえたさらなる検証と改善が今後の大きな課題となることが示唆される.
    \par まず,システム全体の有効性については,\ref{sec:システム全体の有効性}節で詳述したとおり,実データを用いたシミュレーションにより再配置コストを評価指標として検証を行った.その結果,提案した数理最適化ベースの割り当てモデルが,自転車数をスケールアップした場合に特に有意な効果を示すことが明らかとなった.また,サービス提供開始から一定時間経過後においても,再配置コストが低い水準で推移することがかくにんできた.これらの結果は,提案システムが長時間運用や大規模展開に適している可能性を示唆している.
    \par 次に,CtoC化によるメリットとデメリットについては,\ref{sec:CtoC化によるメリットとデメリット}節で述べたとおり,サービス利用者と自転車提供者の双方に利点が存在する.利用者側では,ドックレスかつ乗り捨て可能なシステムにより,移動の自由度が向上し,都市部だけでなく地方でもサービス利用が可能となる潜在性がある.一方,提供者側では,自転車のアイドルタイムを有効活用することで収益を得られる利点がある.しかしながら,待ち時間の発生や盗難リスクといった問題点も指摘でき,これらへの対策が必要不可欠である.
    \par 技術的課題と解決策については,\ref{sec:技術的課題と解決策}節で検討したように,数理最適化モデルとスマートロックの連携が最優先の課題として挙げられる.また,支払処理機能や評価・レビュー機能の実装もサービスの信頼性向上のために重要である.さらに,大規模なシミュレーションを行う際の計算リソースの制約も明らかとなったため,GPU環境の活用などによる効率的な計算資源の確保が求められる.
    \par 社会的インパクトと倫理的考慮に関しては,\ref{sec:社会的インパクトと倫理的考慮}節で述べたとおり,不法駐輪問題や自転車の放置などの社会的課題が懸念される.また,自転車提供者のプライバシー保護や盗難防止策の徹底など,倫理的観点からの配慮も重要である.これらの課題に対しては,規制当局やコミュニティとの共同による適切なルール作りや,技術的なソリューションの導入が求められる.
    \par 加えて,本システムは新たなモビリティサービスとして利用者の利便性向上に寄与するだけでなく,地域経済の活性化や環境負荷の低減といった広範な社会的効果も期待できる.
    \par 総合的に,本研究の提案システムは,新しいモビリティサービスの形態として有望である一方,多方面にわたる課題解決が必要であることが示唆された.今後は,技術的課題の克服とともに,社会的・倫理的側面における対策を講じることで,持続可能なサービスの実現を目指す必要がある.また,数理最適化モデルのさらなる改良や大規模データでの検証を進めることで,システムの実用性を高めていくことが期待される.
  
  \subsection{システム全体の有効性}
    \label{sec:システム全体の有効性}
      \par まずはじめに,本研究全体を通して取り組んだ課題やその目的について改めて簡単にまとめる.本研究では,比較的偏りが少なく,各地に点在している個人所有の自転車に焦点を置き,そのリソースを有効活用して社会全体としてのモビリティ体験を向上させることを目的として,個人間で自転車をシェアリング可能とするCtoCでドックレスなシェアサイクルシステムを提案し,実装した.
      \par そのため,システムを実装後の実データを用いたシミュレーションの際には,自転車の散らばり具合を再配置コストと命名し,それを最重要評価指標として設定し,評価を行った.その結果の1つが図\ref{fig:モデル別再配置コストの時間経過比較}である.
      \par しかし,前述している通り,検証時の状態は,ニューヨーク市内に自転車を10台ランダム配置した状態であり,明らかにユーザーリクエストによる需要に対して自転車の供給が足りていない状況であり,再配置コストの推移が現実的なものであるかどうか判断しかねる結果となっていた.実際に,図\ref{fig:自転車の割り当て成功率とステータスの変化}から分かる通り,割り当て成功率が高くない.ニューヨーク市内の7万以上のタクシーデータに対して10台の自転車しか用意していないからである.
      \par そこで,初期配置する自転車を50台にスケールアップして同様にシミュレーションを行ってみた結果が図\ref{fig:50台の時の再配置コストの時間経過比較}であった.ただし,初期配置する自転車を50台にスケールアップした場合でも,やはり7万以上のユーザーリクエストに対しては非常に不足している状態であるため,割り当て成功率の上昇には期待できなかった.それでも,スケールアップ前と後で再配置コストの推移の結果を見比べてみると,その違いが顕著に表れるようになった.24時間継続してサービスを提供した後の再配置コストは,バッチ最適化割り当てモデルを除いた全てのモデルが同じような値に収束していることが分かる.一定期間経過後は,バッチ最適化割り当てモデルが他のモデルと比較し,低い水準で再配置コストが推移している.
      \par さらに,初期配置する自転車を100台にスケールアップして同様にシミュレーションを行ってみた結果が図\ref{fig:100台の時の再配置コストの時間経過比較}であった.この場合でも変わらずバッチ最適化割り当てモデルが他のモデルと比較し,低い水準で再配置コストが推移していることが読み取れるが,自転車の初期配置数を10台から50台へとスケールアップした場合と比べて,50台から100台へのスケールアップは,別の割り当てモデルとの差が倍率としてそれほど開くことはなかった.
      \par これらの結果から,24時間継続してサービスを提供した場合,本研究で提案した数理最適化ベースの割り当てモデルが有意であることが分かる.また,基本的には,自転車数をスケールアップした場合の方が,再配置コストにより有意性が増しているため,スケールメリットを持つ割り当てモデルであることが期待される.しかし,より細かなスケール幅で自転車数を増減させて検証し,詳細な結果をまとめる必要があるが,ある一定数の自転車数を超えるとそれ以上のスケールメリットは享受できない可能性も示唆される.
      \par さらに,自転車を10台初期配置した場合の結果である図\ref{fig:モデル別再配置コストの時間経過比較}から,サービス提供開始後8時間までは,逐次最適化割り当てモデルとバッチ最適化割り当てモデルとで再配置コストに大きな違いは見られなかった.逐次最適化割り当てモデルは文字通り,1リクエストずつ処理し,自転車の割り当てを行うため,ユーザーに待ち時間を発生させないというメリットがある.そのため,CtoC事業者側が状況によってより最適なモデルを選択できるという考察もできた.しかし,50台にスケールアップした状況下においていは,逐次最適化割り当てモデルはサービス提供開始直後から再配置コストが比較的急激な上昇を見せているため,その限りではないことが読み取れる.より大規模な自転車データを準備した状態で検証し,それらの仮説を明らかにしていく必要がある.
      \par 数理最適化モデルの定式化の観点からも考察したい.数理最適化ベースのマッチングモデルを設計する際に,その目的関数を式(\ref{equ:目的関数})と定義した.この式は,「ユーザーが,割り当てられた自転車に乗って移動した後の自転車とその自転車の所有者までの距離を最小化」することと,「可能な限り多くのユーザーに自転車を割り当てる最大化」することに対するトレードオフを重みを掛け合わせることで調整している.本研究では,簡単のため,この重みを1に固定して実装しているが,この重みの最適解を探る検証も不可欠である.
      \par また,シミュレーション検証において,本研究ではモデルの検証のため,ニューヨーク市のタクシーデータを用いたが,自転車に比べてタクシーは移動距離が大きくなると考えられる.つまり,実際のCtoCシェアサイクルサービスを想定した場合,タクシーよりも移動距離が短く,範囲も限定されることから,シミュレーションを行なった結果よりもより小さい値でリバランスコストが推移する可能性も十分考えられる.
      \par さらに,今後実証実験の環境を整えることができれば,実際の運用環境から得られるリアルタイムデータを活用することで,より柔軟なパラメータ調整やアルゴリズムの改良が期待できると考えられる.
      
  \subsection{CtoC化によるメリットとデメリット}
    \label{sec:CtoC化によるメリットとデメリット}
      \par 本研究にて提案したシステムによってCtoC化を実現することによるメリットは,サービス利用者と自転車提供者の双方にある.サービス利用者側においては,乗り捨て可能なシステムを前提としていることから移動自由度が大きく上昇することが期待される.また,従来のシェアサイクルサービスが豊富に展開されいている都市部に限らず,地方でもサービスを利用することができる可能性を秘めている.
      \par 自転車提供側においては,自身の自転車を利用していない時間帯にその自転車を有効利用できる点にある.アイドルタイムが発生する対象に対してシェアリングすることで,提供者側は,いわゆる不労所得を得ることができる.また,シェアリングした自転車が乗り捨てされる場合,一時的に所有者のもとに返却されないものの,数理最適化アルゴリズムによる割り当て処理を行っていることから,所有者の元に自動的に返却される,もしくはその近くに駐輪される可能性が高いため,所有者が利用したい場合もスムーズに対応できることが期待される.
      \par 一方で,本研究にて提案したシステムによってCtoC化を実現することによるデメリットも存在する.サービス利用者側においては,待ち時間の発生にある.図\ref{fig:モデル別再配置コストの時間経過比較}や図\ref{fig:50台の時の再配置コストの時間経過比較}を参照してみると分かる通り,逐次最適化割り当てモデルよりもバッチ最適化割り当てモデルの方が再配置コストが低い.本研究の場合,バッチ幅は1分間としており,1分間のユーザーリクエストをストックすることとなる.ある1分間に対して,毎秒のリクエストが送られてくる確率が同様に確からしいとすると,ユーザーはリクエストを行ってからレスポンスが返ってくるまでに平均して30秒間待機する必要があることになる.
      \par 実験的にストックする時間を1分として実装しているが,ユーザーの平均待機時間が30秒であることは,ユーザー体験として良いとは言えない.さらに,ユーザーリクエスト数や提供される対象の自転車数が大規模であった場合,数理最適化割り当て処理自体にも処理時間が発生し,30秒以上待機する必要がある.一方で,システム全体としては,リクエストをストックする時間に応じて数理最適化の旨味が増し,再配置コストが低い結果を得ることが期待される.このように,ユーザー体験とシステム効率のトレードオフを探るにあたって,リクエストをストックする時間幅の最適解も探る必要がある.
      \par 自転車提供側においては,自身の自転車が盗難される危険性がある点がデメリットになり得るだろう.自転車の鍵はスマートロックで管理し,対象のユーザーはそのスマートロックを解錠できることになる.悪意のあるユーザーに利用されてしまった場合,簡単に自転車を盗難できる状況が発生する.利用する前のユーザー登録時などに厳正な個人認証を行い,盗難を働きかけたとしてもすぐに特定できる仕組みや抑止力が必要であると考えらえる.
      \par さらに,本システムの普及は,利用者の移動の利便性向上に加えて地域経済の活性化や環境負荷の低減といった社会的メリットをもたらす可能性がある一方,地域ごとの交通ルールや文化に即した運用方法の検討が不可欠となる.
      
  \subsection{技術的課題と解決策}
    \label{sec:技術的課題と解決策}
      \par 技術的課題として最もコアな点は,自転車割り当てモデルとスマートロックが未結合である点が挙げられる.APIを実装したものの,構築した割り当てモデルを利用するためのインターフェースにしかすぎず,割り当て処理を行った結果からスマートロックと連携し,解錠及び施錠の操作を行うまでの実装には至らなかった.一連のシステムとして大成するためにはこの連携は急務の課題である.
      \par また,他にも,設計したものの実装まで至ることはできなかった機能として,支払処理機能や評価・レビュー機能が挙げられる.これらも,スマートロック連携よりはプライオリティが下がるものの,必要不可欠な機能の1つであるためAPIのマイクロサービスの1つとして実装してく必要がある.
      \par 構築した自転車割り当てモデルを用いてシミュレーション実験を行う際にも技術的課題に直面した.計算リソースの不足である.大規模なケースとして,ニューヨーク市内に650台の自転車を初期配置したシチュエーションでのシミュレーションを試みたが,かなりの時間的計算コストを要し,結果を取得するまでに至らなかった.CPU環境を用いた実験環境を構築したことが,時間的計算コストを要した一因として考えられるため,GPU環境を用いた実験環境の構築と,GPUリソースを最大限生かすことのできるシミュレーションのソースコード修正が必要になると考えられる.
      \par さらに,システム全体のセキュリティ対策の強化や不正利用防止のための監視機能の充実も,今後の実装において重要な検討項目として挙げられる.
      
  \subsection{社会的インパクトと倫理的考慮}
    \label{sec:社会的インパクトと倫理的考慮}
      \par 本研究にて提案したシステムをサービスとして社会実装するとなった際に考え得る社会的影響として,自転車の放置問題が挙げられる.ドックレスでどこでも乗り捨て可能なシステムとしているからである.実際に既存のドックレスシェアサイクルサービスでも駐輪場の不足や,それによって自転車が歩行者用通路にはみ出して駐輪されている問題も散見される.本研究のサービス設計では,そうした課題がより顕著に表れる可能性も十分に想定される.
      \par また,実現可能性についてもまだまだ課題がある.本研究で提案した数理最適化ベースの割り当てモデルの設計を行う際に,前提条件として,自転車はどこでも乗り捨てが可能であることとし,乗り捨てによる法的な側面の課題はモデル構築において考慮しないと言及している.しかし,実社会でサービス提供を行う際には最もネックになる部分であると考えられる.
      \par さらに,サービスを利用するユーザーのプライバシー保護にも重点を置かなければならない.特に,自転車提供側はその自転車のあるべき駐輪場の位置情報を共有するべきであるが,それがそのユーザーの住所である可能性も大いに考えられ,どのように自転車利用者にそれを公開するのか,安全な方法を検討しなければならない.
      \par \ref{sec:CtoC化によるメリットとデメリット}節でも述べた盗難に関する課題やここで述べた不法駐輪等の公共道路に関する課題・ユーザーのプライバシー保護などの倫理的に考慮しなければならない点は,より詳細を詰めて考える必要がある.
      \par さらに,利用者間の信頼性向上や情報公開の透明性を確保する仕組みの導入,及び地域ごとの法制度や規制に柔軟に対応するためのガイドライン整備も,今後のサービス実現に向けた重要な課題として認識される.
