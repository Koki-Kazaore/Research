% 第8章
\section{結論と今後の展望}
  \label{sec:結論と今後の展望}  
    \par 本研究では,CtoCドックレスシェアサイクルの実現を目指し,その可能性についての議論や,具体的なビジネスモデルの検討,技術的要件の抽出やシステムの実装を行なった.全章を俯瞰すると,まず関連研究の整理を行った.ここではシェアサイクルの現状や課題,研究で必要となるスマートロックやマッチングアルゴリズムなどの技術的な研究動向についても多くの文献を参考にし,本研究の立ち位置を明らかにすることにも繋がった.その後,CtoCシェアサイクルのサービス設計とビジネスモデルを定義し,本研究の対処となるドメイン領域への理解を深めたり,既存サービスとの差別化を行ったりした.
    \par 続く章にて,システムの設計や実装を行なった.設計では要件定義に始まり,サービスとして価値提供するにあたって将来的に必要となる機能も考慮しながら必要事項をまとめた.設計内容を実装するにあたっては,本論文を読んでシステム構築を再現することができるよう可能な限り具体的に手順を紹介した.構築したモデルに対しては,実際のニューヨーク市のタクシートリップデータを利用しシミュレーション実験を行なった.サービス提供後の自転車の再配置コストを最重要評価指標に設定し評価を行なった結果,本研究で構築した自転車割り当てモデルが最も有意な結果を得ることができた.またその結果に対する考察も続く章にて行なった.
    \par 本研究全体を通して振り返ると,CtoCドックレスシェアサイクルの実現には至らなかったものの,そのコア要素となる自転車の割り当てモデルは十分な性能を発揮するモデルを構築できたと言える.また,既存のBtoC型シェアサイクルサービスやそれらに関わる研究が中心の中で,CtoC型に特化したサービス提案や数理最適化をベースとした自転車割り当てモデルの構築,それに伴う自転車のスマートロックのプロトタイプ提案を行なった点も本研究の貢献である.
    \par さらに,本研究のアプローチは,ユーザー行動や実際の運用状況を反映したシミュレーションに基づいており,今後はリアルタイムデータとの連携を進めることで,よりダイナミックなシステム運用の実現が期待される.また,システム全体の効率性向上のためのアルゴリズムのチューニングや分散処理の導入,さらにはクラウド環境を活用した大規模シミュレーションの実施が今後の重要な課題となるであろう.
    \par それでも,今後の考慮していかなければならない課題は山積みである.例えば,地理的制約を追加する必要がある点である.本研究で構築した自転車の割り当てモデルのアルゴリズムでは,障害物の無い平面上として処理を行なっている.しかし,現実世界では多様な障害物が存在する.あるユーザーからのリクエストに対して直線距離的にすぐ近くに適切な自転車が駐輪されていたとしても,ユーザーと自転車の間は河川で隔てられており,自転車の場所までユーザーが到達するためには,回り道をして橋を渡らなければならないなどの状況がその一例である.この場合,ユーザー体験が非常に悪化することになりかねない.対処方法の一案としては,サードパーティ製のAPIを利用し,二地点間の道のりによる距離を取得した上で割り当て処理を行う方法が考えられる.現実世界における地理的制約は複雑であるもののクリティカルな課題であるため,CtoCドックレスシェアサイクルの実現のためには避けて通ることはできないだろう.
    \par また,時間制約を追加しなければならない点もまだ残されている課題である.本研究にて行なったシミュレーションでは,初期配置している自転車の全てが,24時間オーナが利用せず,他のユーザーが利用していない限りシェアリングの対象となることを前提として行われている.しかし実際には,自転車オーナは12時間後に自身で利用する予定があるかもしれないし,サービス提供開始後数時間経ってからシェアリングの対象として貸し出すことができるようになるかもしれない.シミュレーション自体はシンプルであることに意義があるため,過剰な考慮事項になるかもしれないが,シチュエーションとしては対処できなければならない項目であるため,これも,CtoCドックレスシェアサイクルの実現のためには避けて通ることはできない.
    \par 他にも,ユーザーの選好を元にした優先度や特定の条件を反映する制約の追加も必要である.例えば,ユーザーがクロスバイクを利用したいと考えているのか,もっとシンプルな自転車で事足りるかなどの条件で割り当てる自転車を考慮しなければならない点がその一例である.
    \par 今後の展望として,上記の課題に対するアプローチや法的側面における課題などをクリアし,実証実験等に取り組むことができればより面白い研究に発展していくことができると期待している.CtoCのシェアリングであることを最大限活かし,オーナが自由に,自転車の利用のための時間単価を設定したり,人気の自転車の場合には,ユーザー側が価格を押し上げて提案する方式や機能があっても面白いかもしれない.本研究をきっかけとして,CtoCドックレスシェアサイクルが実現できることを切に願っている.モビリティ体験のさらなる向上に思いを馳せて.
